\documentclass[a4paper,10pt]{article}
\usepackage[utf8]{inputenc}
\usepackage{fancyhdr, float, graphicx, caption}
\usepackage{amsmath, amssymb}
\usepackage{bm}
\usepackage[margin=1in]{geometry}
\usepackage{multicol}
\usepackage{proof}
\usepackage{titlesec} 

\setlength{\inferLineSkip}{4pt}

\titleformat{\subsection}[runin]
  {\normalfont\large\bfseries}{\thesubsection}{1em}{}	
\titleformat{\subsubsection}[runin]
  {\normalfont\normalsize\bfseries}{\thesubsubsection}{1em}{}


\pagestyle{fancy}
\renewcommand{\figurename}{Figura}
\renewcommand\abstractname{\textit{Abstract}}

\fancyhf{}
\fancyhead[LE,RO]{\textit{Intérprete de Cálculo Lambda}}
\fancyfoot[RE,CO]{\thepage}

%%%%%%%%%%%%%%%%%%%%%%%%%%%%%%%%%%%%%%%%%%%%%%%%%%%%%%%%%%%%

\title{
	%Logo UNR
	\begin{figure}[!h]
		\centering
		\includegraphics[scale=1]{unr.png}
		\label{}
	\end{figure}
	% Pie Logo
	\normalsize
		\textsc{Universidad Nacional de Rosario}\\	
		\textsc{Facultad de Ciencias Exactas, Ingeniería y Agrimensura}\\
		\textit{Licenciatura en Ciencias de la Computación}\\
		\textit{Análisis de Lenguajes de Programación}\\
	% Título
	\vspace{30pt}
	\hrule{}
	\vspace{15pt}
	\huge
		\textbf{Intérprete de Cálculo Lambda}\\
	\vspace{15pt}
	\hrule{}
	\vspace{30pt}
	% Alumnos/docentes
	\begin{multicols}{2}
	\raggedright
		\large
			\textbf{Alumnos:}\\
		\normalsize
			CRESPO, Lisandro (C-6165/4) \\
			MISTA, Agustín (M-6105/1) \\
			$\;$ \\
			$\;$ \\
	\raggedleft
		\large
			\textbf{Docentes:}\\
		\normalsize
			JASKELIOFF, Mauro\\
			RABASEDAS, Juan Manuel\\
			SIMICH, Eugenia\\
			MANZINO, Cecilia\\
	\end{multicols}
}
%%%%%%%%%%%%%%%%%%%%%%%%%%%%%%%%%%%%%%%%%%%%%%%%%%%%%%%%%%%
\begin{document}
\date{22 de Septiembre de 2015}
\maketitle

\pagebreak
%----------------------------------------------------------
\subsection*{Ejercicio 2.1.} 
	\emph{Definimos la gramática extendida del $\lambda$-cálculo.}
	\\
	\begin{alignat*}{3}
		\langle atom \rangle \; &:= \; \langle var \rangle\; | \;\langle number \rangle\; | \;{\textbf (} \; \langle term \rangle \; {\textbf )}\\[0.5em]
		\langle ids \rangle \; &:= \; \langle var \rangle\ ( \; \epsilon\; | \; \langle ids \rangle\; )\\[0.5em]
		\langle abs \rangle \; &:= \; \lambda \;\langle ids \rangle\; {\textbf .}\; \langle term \rangle \\[0.5em]
		\langle notAbs \rangle \; &:= \; \langle atom \rangle\; | \;\langle notAbs \rangle \; \langle notApp \rangle \\[0.5em]
		\langle notApp \rangle \; &:= \; \langle atom \rangle\; | \;\langle abs \rangle \\[0.5em]
		\langle term \rangle \; &:= \; \langle abs \rangle\; | \;\langle notAbs \rangle \\[0.5em]
	\end{alignat*}
%----------------------------------------------------------
\subsection*{Ejercicio 2.2.}
	\emph{Eliminamos la recursión a izquierda de la gramática planteada en el item anterior.}
	\\
	\begin{alignat*}{3}
			\langle atom \rangle \; &:= \; \langle var \rangle\; | \;\langle number \rangle\; | \;{\textbf (} \; \langle term \rangle \; {\textbf )}\\[0.5em]
		\langle ids \rangle \; &:= \; \langle var \rangle\ ( \; \epsilon\; | \; \langle ids \rangle \; )\\[0.5em]
		\langle abs \rangle \; &:= \; \lambda \;\langle ids \rangle\; {\textbf .}\; \langle term \rangle \\[0.5em]
		\langle notAbs \rangle \; &:= \; \langle atom \rangle\; \langle notAbs' \rangle \\[0.5em]
		\langle notAbs' \rangle \; &:= \; \langle notApp \rangle \; \langle notAbs' \rangle \; |\ \epsilon  \\[0.5em]
		\langle notApp \rangle \; &:= \; \langle atom \rangle\; | \; \langle abs \rangle \\[0.5em]
		\langle term \rangle \; &:= \; \langle abs \rangle\; | \;\langle notAbs \rangle \\[0.5em]
	\end{alignat*}
\\
\\
\vspace{\fill}
\begin{multicols}{2}
	\hrule
	\vspace{5pt}
	CRESPO, Lisandro \\
	\linebreak

	\hrule
	\vspace{5pt}
	MISTA, Agustín \\
\end{multicols}

\end{document}
